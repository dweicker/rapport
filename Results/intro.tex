\chapter{Results and discussion}
In this chapter, we will put the algorithms presented in the theory chapter to the test and present the results. The implementation of the algorithms consists of a code written in C of more than 7000 lines which leverages the p4est library (see the implementation chapter for information) for the mesh generation and refinement as well as the Lapack library (a widely used linear algebra library, see \textcolor{red}{ref here!}) to solve linear systems and diagonalize matrices. We will look at the experimental results and compare them to the theory developed. The code written can in principle handle any order of interpolation and any geometry but in practice most tests have been performed for $p=2,4,6,8$ and on $\Omega = [-1;1]^2$ which allowed us to have analytic solutions to the problems we investigated. 

This chapter is divided into several sections. The first looks at the geometric multigrid preconditioner. As explained in the theory chapter, it cannot be used as the only preconditioner since it has a kernel but we can use it as an iterative solver for $p=1$. This is done to verify an important property of the geometric multigrid methods : the h-independent convergence. In this section, we also look at the influence of hanging nodes on the solver.

The second section focus on the preconditioned conjugate gradients with only the fine preconditioner. This aims at looking at the properties of the overlapping Schwarz preconditioner presented in the theory chapter. We will first test it on regular elements, where the preconditioner is optimal, then look at what happens when we have a mesh with distorted elements. We will afterwards move on to non conforming meshes and look at the influence of having hanging nodes on the fine preconditioner. Finally, we will increase the degree of the interpolation and observe how the number of iterations of PCG evolves. 

The third section looks at the PCG with the two scale preconditioner : both the coarse grid correction computed by the multigrid solver and the fine scale correction computed by the overlapping Schwarz method. The addition of the coarse scale preconditioner should provide a convergence independent of the number of quadrants. As before, we will first test the algorithms on a regular mesh, where it should perform very well. Then, we will see how it fares on meshes with distorted elements and what happens when we increase the distortion. We will thereafter consider non conforming meshes obtained through adaptive mesh refinement. We will check that here also we have a convergence that is independent of the number of quadrants and how the number of iterations of PCG compares to the case of conforming meshes. Then, we increase the degree of the interpolation on conforming meshes and look at what happens to the number of iterations. Finally, we emphasize the importance of being able to handle higher orders of interpolation. Given a wanted accuracy on the $L^2$-norm of the error, we present the best polynomial order to use for a given problem and a given mesh. 

The chapter ends on a partial conclusion which summarize the results presented before.

  