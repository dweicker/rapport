\section{Conclusion}

Let us now summarize the results given in this chapter. 

We first looked at the properties of our geometric multigrid solver. The key property of h-independent convergence has been observed for a large variety of parameters $\nu_1$ and $\nu_2$ and various meshes that were regular or distorted. The presence of hanging nodes has no influence whatsoever on the convergence of the geometric multigrid method. 

We then moved on to the PCG with only the fine scale preconditioner. Several observations were made. We saw that, even on a regular mesh where the preconditioner is optimal, increasing the number of quadrants increased the number of iterations needed. That was predicted since increasing the number of quadrants decreases the mesh size. We also noticed that we needed more iterations when the degree of interpolation $p$ grew. That was explained by the fact that the size of the overlap diminishes when $p$ increases. We then move on to meshes with distorted quadrants and saw that we needed more iterations than in the regular case. It was to be expected since the fine preconditioner was designed to work best on quadrants aligned with the axes. Afterwards, we introduced non conforming meshes. The presence of hanging nodes also increases the number of iterations. That was explained by the fact that we do not restrict the residual exactly in the overlaps. When we increase the degree of the interpolation on a non conforming, the number of iterations increases as in the conforming case. 

The coarse grid correction was then added in the preconditioner. We first tested the PCG with the two scale preconditioner on conforming meshes and saw that we obtained the h-independence thanks to the coarse preconditioner : the number of iterations stayed constant whatever the number of quadrants and the mesh size. That was true for both meshes with regular quadrants and meshes with distorted quadrants. We can however note that the number of iterations needed was greater in the case of distorted quadrants. Again, this was to be expected since the fine preconditioner works better on quadrants aligned with the axes. We then experimented on non conforming meshes. We saw that changing the absolute or the relative number of hanging nodes in the mesh had no real impact of the number of iterations. That being said, we saw that we needed more iterations when we had hanging nodes (even a few). We also looked at what happened when we increased the degree of the interpolation. As before, it increases with the degree $p$. The same remarks as before also apply here. 

Finally, we picked a problem and a mesh and looked at the best degree of interpolation to solve this problem, i.e. the degree which allowed us to get the solution within a given tolerance in the $L^2$-norm the quickest. We saw that at first, for a low tolerance, a low degree is best since then the fine preconditioner is cheap to compute and we can have a lot of degrees of freedom. The number if iterations of PCG is also lower for lower degrees. Then, as we tighten the tolerance, it becomes more and more attractive to use higher degrees since they allow us to keep the number of degrees of freedom down and still get a good enough approximation to be within the tolerance. The increases in the number of iteration for higher degrees is compensated by the fact that we have a lot less degrees of freedom. This means that for a given tolerance, we have a trade-off between the number of iterations of PCG and the number of degrees of freedom. 